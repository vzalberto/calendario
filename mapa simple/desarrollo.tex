\section{Desarrollo}

A continuación, agregaremos una caja de búsqueda. 
\\
Antes que otra cosa, debemos solicitar la libería de lugares a Google. Basta incluirla como parámetro en la solicitud del atributo <script>:

\begin{lstlisting}
 <script src="https://maps.googleapis.com/maps/api/js?v=3.exp&signed_in=true&sensor=true&libraries=places"></script>
 \end{lstlisting}

 Para obtener la ubicación del usuario, recurrimos al método getCurrentPosition, de la interfaz navigator.geoLocation. Está definido así:

\begin{lstlisting}
 interface Geolocation { 
   void getCurrentPosition(PositionCallback successCallback,
                           optional PositionErrorCallback errorCallback,
                           optional PositionOptions options);

   long watchPosition(PositionCallback successCallback,
                      optional PositionErrorCallback errorCallback,
                      optional PositionOptions options);

   void clearWatch(long watchId);
 };

 callback PositionCallback = void (Position position);

 callback PositionErrorCallback = void (PositionError positionError);
 \end{lstlisting}

 de manera que debemos definir una función para el manejo del objeto Position position que nos será devuelto. Escribimos la siguiente función:

 \begin{lstlisting}
 function updateMapCenter(Position position){
  userPosition = position;
  map.setCenter(new google.maps.LatLng(position.coords.latitude, position.coords.longitude));
  map.setZoom(15);
}
\end{lstlisting}

que, como se puede ver, realiza 3 sencillas acciones: actualizar una variable global con la posición geográfica del usuario, actualiza el centro del mapa previamente inicializado y acerca el zoom.



Ahora definiremos un marcador para la ubicación del usuario. Solo debemos invocar al constructor del objeto Marker y añadir un objeto de opciones. En este caso se especifican la ubicación (requerida) que será la misma que ya tenemos guardada, la referencia al objeto Map, una animación y un texto que se mostrará cuando el cursor se encuentre sobre el marcador. La función es llamada al final de updateMapCenter que definimos en el inciso anterior.

\begin{lstlisting}
function setUserMarker(){
  options = {
    map: map,
    position: new google.maps.LatLng(userPosition.coords.latitude, userPosition.coords.longitude),
    animation: google.maps.Animation.DROP,
    title: '( ͡° ͜ʖ ͡°) te observo'
  }
  marker = new google.maps.Marker(options);
}
\end{lstlisting}

{{añadir screen}}

Por último, para trazar rutas es necesario recurrir al servicio de direcciones de google. Declaramos las siguientes variables globales:

\begin{lstlisting}

var directionsDisplay;
var directionsService = new google.maps.DirectionsService();

\end{lstlisting}

y el siguiente método:

\begin{lstlisting}
function calcRoute(end) {
  var start = new google.maps.LatLng(userPosition.coords.latitude, userPosition.coords.longitude);
  var request = {
      origin:start,
      destination:end,
      travelMode: google.maps.TravelMode.DRIVING
  };
  directionsService.route(request, function(response, status) {
    if (status == google.maps.DirectionsStatus.OK) {
      directionsDisplay.setDirections(response);
    }
  });
}
\end{lstlisting}

que hace un llamado al servicio de rutas de manera asíncrona. En cuanto se recibe el código "OK", se pasa el objeto de respuesta al conveniente objeto directionsDisplay que lo muestra en pantalla. El llamado que hacemos especifica punto de salida (coordenadas del usuario), punto destino (lo que escribe el usuario mediante el HTML) y el tipo de viaje, en este caso, coche.

https://developers.google.com/maps/documentation/javascript/directions
http://stackoverflow.com/questions/9181862/onsubmit-refresh-html-form